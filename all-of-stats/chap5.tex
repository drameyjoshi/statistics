\documentclass{article}
\usepackage{amsmath, amssymb, amsfonts, amsthm}
\usepackage{mathtools}
\usepackage{hyperref}
\usepackage{mathabx}

\newcommand{\son}{\mathbb{N}}
\newcommand{\soi}{\mathbb{Z}}
\newcommand{\soq}{\mathbb{Q}}
\newcommand{\sor}{\mathbb{R}}
\newcommand{\soc}{\mathbb{C}}
\newcommand{\op}{\prime}
\newcommand{\tp}{\prime\prime}
\newcommand{\td}[2]{\frac{d{#1}}{d{#2}}}
\newcommand{\dBer}{\text{Bernoulli}}
\newcommand{\dBin}{\text{Binomial}}
\newcommand{\dGeo}{\text{Geom}}
\newcommand{\dPoi}{\text{Poisson}}
\newcommand{\dUni}{\text{Uniform}}
\newcommand{\dNor}{\mathcal{N}}
\newcommand{\dExp}{\text{Exp}}
\newcommand{\dGam}{\text{Gamma}}
\newcommand{\dBet}{\text{Beta}}

\DeclareMathOperator{\erf}{erf}

\theoremstyle{plain}
\newtheorem{thm}{Theorem}
%\numberwithin{thm}{section}

\theoremstyle{plain}
\newtheorem{prop}{Proposition}
%\numberwithin{prop}{section}

\theoremstyle{definition}
\newtheorem{defn}{Definition}
%\numberwithin{defn}{section}

\theoremstyle{remark}
\newtheorem*{rem}{Remark}

\theoremstyle{plain}
\newtheorem{cor}{Corollary}
%\numberwithin{cor}{section}

%\numberwithin{equation}

\begin{document}
\begin{defn}\label{c5d1}
Let $\{X_n\}$ be a sequence of random variables and $X$ be another random
variable. Then $\{X_n\}$ converges to $X$ in probability, written 
$X_n \stackrel{P}\rightarrow X$, if for a given $\epsilon > 0$, we can find
$N \in \son, \delta \in \sor$ such that 
\[
P(|X_n - X| > \delta) < \epsilon
\]
for all $n \ge N$.
\end{defn}

\begin{defn}\label{c5d2}
Let $\{X_n\}$ be a sequence of random variables each with cdf $F_n$. Let $X$ be
another random variable with cdf $F$. Then $\{X_n\}$ converges to $X$ in 
distribution, written as $X_n\stackrel
{d}\rightarrow X$ if $F_n(t) \rightarrow F(t)$ for all $t$ at which $F$ is 
continuous.
\end{defn} 

\begin{defn}\label{c5d3}
Let $\{X_n\}$ be a sequence of random variables and $X$ be another random
variable. Then $\{X_n\}$ converges to $X$ in quadratic mean, written 
$X_n \stackrel{qm}\rightarrow X$, if for a given $\epsilon > 0$, we can find
$N \in \son$ such that $E(X_n - X)^2 < \epsilon$ for all $n \ge N$.
\end{defn}

\begin{prop}\label{c5p1}
If $X_n \stackrel{qm}\rightarrow X$ then $X_n \stackrel{P}\rightarrow X$.
\end{prop}
\begin{proof}
Fix $\delta > 0$. Then $P(|X_n - X| \ge \delta) = P((X_n - X)^2 \ge \delta^2)$.
Using Markov's inequality, the second term is lesser than $E(X_n - X)^2/\delta^2$.
Thus, we have
\[
P(|X_n - X| \ge \delta) \le \frac{E(X_n - X)^2}{\delta^2}.
\]
Since $X_n \stackrel{qm}\rightarrow X$, for any $\epsilon > 0$, we can find
$N \in \son$ such that $E(X_n - X)^2 < \epsilon \delta^2$ for all $n \ge N$. 
Therefore, for all $n \ge N$, we also have $P(|X_n - X| \ge \delta) < \epsilon$.
\end{proof}

\begin{prop}\label{c5p2}
If $X_n \stackrel{P}\rightarrow X$ then $X_n \stackrel{d}\rightarrow X$.
\end{prop}
\begin{proof}
Fix an $\delta_1 > 0$ and let $x$ be a point of continuity of $F$. Then
\[
F(x) = P(X_n \le x) = P(X_n \le x, X \le x + \delta_1) + 
P(X_n \le x, X > x + \delta_1)
\]
Now for any two event, $A$ and $B$, $P(AB) \le P(A)$. Therefore,
\[
F_n(x) \le P(X \le x + \delta_1) + P(X_n \le x, X > x + \delta_1) = 
F(x + \delta_1) + P(X_n \le x, X > x + \delta_1).
\]
Now, $X > x + \delta_1 \Rightarrow X - X_n > x - X_n + \delta_1 > \delta_1$.
Thus, we have
\begin{equation}\label{e1}
F_n(x) \le F(x + \delta_1) + P(X - X_n > \delta_1).
\end{equation}

Likewise,
\begin{eqnarray*}
F(x - \delta_1) &=& P(X \le x - \delta_1) \\
 &=& P(X \le x - \delta_1, X_n \le x) + P(X \le x - \delta_1, X_n > x) \\
 &\le& P(X_n \le x) + P(X \le x - \delta_1, X_n > x) \\
 &\le& F_n(x) + P(X \le x - \delta_1, X_n > x)
\end{eqnarray*}
Now $X_n > x, X \le x - \delta_1 \Rightarrow X_n - X > x - X \ge \delta_1$. Thus,
we have
\begin{equation}\label{e2}
F(x - \delta_1) \le F_n(x) + P(X_n - X \ge \delta_1).
\end{equation}
From equations \eqref{e1} and \eqref{e2} we have
\begin{equation}\label{e3}
F(x - \delta_1) - P(X_n - X \ge \delta_1) \le F_n(x) \le F(x + \delta_1) + 
P(X - X_n > \delta_1)
\end{equation}
Fix an $\epsilon > 0$. Since $X_n \stackrel{P}\rightarrow X$, we can find $N
\in \son$ such that for all $n \ge N$, $P(|X_n - X| > \delta_1) < \epsilon/2$.
Therefore, $-P(|X_n - X| > \delta_1) > -\epsilon/2$ or $F(x - \delta_1) - 
P(X_n - X \ge \delta_1) > F(x - \delta_1) - \epsilon/2$. Therefore, equation \eqref{e3}
gives,
\begin{equation}\label{e4}
F(x - \delta_1) - \frac{\epsilon}{2} < F_n(x) < 
F(x + \delta_1) + \frac{\epsilon}{2}.
\end{equation}
Since $F$ is continuous at $x$, for the same $\epsilon > 0$, we can find 
$\delta_2 > 0$ such that $|F(x) - F(x \pm \delta_2)| \le \epsilon/2$, that is,
$-\epsilon/2 \le F(x) - F(x \pm \delta_2) \le \epsilon/2$ or
\[
F(x \pm \delta_2) -\frac{\epsilon}{2} \le F(x) \le F(x \pm \delta_2) + 
\frac{\epsilon}{2}
\]
or
\begin{equation}\label{e5}
-F(x \pm \delta_2) - \frac{\epsilon}{2} \le -F(x) \le -F(x \pm \delta_2) + 
\frac{\epsilon}{2}
\end{equation}
Let $\delta = \min(\delta_1, \delta_2)$. Then equations \eqref{e4} and \eqref{e5}
can be written as 
\begin{eqnarray*}
F(x - \delta) - \frac{\epsilon}{2} &<& F_n(x) < F(x + \delta) + \frac{\epsilon}{2} \\
-F(x - \delta) - \frac{\epsilon}{2} &\le& -F(x) \le -F(x \pm \delta) + \frac{\epsilon}{2}.
\end{eqnarray*}
Adding them we get $-\epsilon \le F_n(x) - F(x) \le \epsilon$, that is, $|F_n(x)
- F(x)| \le \epsilon$.
\end{proof}

\end{document}
