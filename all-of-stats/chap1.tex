\documentclass{article}
\usepackage{amsmath, amssymb, amsfonts, amsthm}
\usepackage{mathtools}
\usepackage{hyperref}
\usepackage{bm}
%\usepackage{mathabx}

\newcommand{\son}{\mathbb{N}}
\newcommand{\soi}{\mathbb{Z}}
\newcommand{\soq}{\mathbb{Q}}
\newcommand{\sor}{\mathbb{R}}
\newcommand{\soc}{\mathbb{C}}
\newcommand{\op}{\prime}
\newcommand{\tp}{{\prime\prime}}
\newcommand{\td}[2]{\frac{d{#1}}{d{#2}}}
\newcommand{\dBer}{\text{Bernoulli}}
\newcommand{\dBin}{\text{Binomial}}
\newcommand{\dGeo}{\text{Geom}}
\newcommand{\dPoi}{\text{Poisson}}
\newcommand{\dUni}{\text{Uniform}}
\newcommand{\dNor}{\mathcal{N}}
\newcommand{\dExp}{\text{Exp}}
\newcommand{\dGam}{\text{Gamma}}
\newcommand{\dBet}{\text{Beta}}
\newcommand{\dMul}{\text{Multinomial}}

\renewcommand{\vec}[1]{\bm{#1}}

\DeclareMathOperator{\erf}{erf}
\DeclareMathOperator{\var}{Var}
\DeclareMathOperator{\cov}{Cov}

\theoremstyle{plain}
\newtheorem{thm}{Theorem}
%\numberwithin{thm}{section}

\theoremstyle{plain}
\newtheorem{prop}{Proposition}
%\numberwithin{prop}{section}

\theoremstyle{definition}
\newtheorem{defn}{Definition}
%\numberwithin{defn}{section}

\theoremstyle{remark}
\newtheorem*{rem}{Remark}

\theoremstyle{plain}
\newtheorem{cor}{Corollary}
%\numberwithin{cor}{section}

%\numberwithin{equation}


\begin{document}
\begin{enumerate}
\item[1.] Solved in the Python notebook for this chapter.

\item[2.] Quite straightforward.

\item[3.] Given that $\Omega$ is a sample space and $A_1, A_2, \ldots$ are
events. Let
\begin{eqnarray*}
B_n &=& \bigcup_{k=n}^\infty A_k \\
C_n &=& \bigcap_{k=n}^\infty A_k
\end{eqnarray*}

Clearly, $B_{n+1} \cup A_{n} = B_{n}$ so that $B_n \supseteq B_{n+1}$ for all $n
\ge 1$, or $B_1 \supseteq B_2 \supseteq \ldots$. Likewise $C_{n+1} \cap A_{n} = 
C_n$ so that $C_{n+1} \supseteq C_n$ or that $C_1 \subseteq C_2 \subseteq \ldots$.

Now,
\[
\omega \in \cap_{k=1}^\infty B_k \Rightarrow \omega \in B_k \forall k \in \son
\Rightarrow \omega \in \bigcup_{j=k}^\infty A_j \forall k \in \son \Rightarrow
\omega \in A_j \forall j \in \son
\]
Conversely, suppose that $\omega \in A_k$, for all $k \in \son$. Then $\omega \in
B_n$ for all $n$ and hence $\omega \in \cap_{n=1}^\infty B_n$.

On the other hand
\[
\omega \in \cup_{k=1}^\infty C_k \Rightarrow \exists k \in \son, \omega \in C_k
\Rightarrow \omega \in \cap_{j=k}^n A_j \Rightarrow \omega \in A_j \forall j
\ge k.
\]
Conversely, if $\omega \in A_k$ for all $k \ge N$ then $\omega \in C_k$ for all
$k \ge N$ and hence $\omega \in \cup_{k=1}^\infty C_k$.

\item[4.] Let $I$ be an arbitrary index set and let
\[
x \in \left(\bigcup_{i \in I}A_i\right)^c \Rightarrow x \notin \bigcup_{i \in I}A_i
\Rightarrow x \notin A_i \forall i \Rightarrow x \in A_i^c \forall i \Rightarrow 
x \in \bigcap_{i\in I}A_i^c,
\]
so that
\begin{equation}\label{e1}
\left(\bigcup_{i \in I}A_i\right)^c \subset \bigcap_{i\in I}A_i^c.
\end{equation}
If,
\[
x \in \cap_{i\in I}A_i^c \Rightarrow x \in A_i^c \forall i \Rightarrow
x \notin A_i \forall i \Rightarrow x \notin \bigcup_{i \in I}A_i
\Rightarrow x \in \left(\bigcup_{i \in I}A_i\right)^c,
\]
so that
\begin{equation}\label{e2}
\bigcap_{i\in I}A_i^c \subseteq \left(\bigcup_{i \in I}A_i\right)^c.
\end{equation}
From \eqref{e1} and \eqref{e2},
\[
\bigcap_{i\in I}A_i^c = \left(\bigcup_{i \in I}A_i\right)^c.
\]

If we make the substitution $A_i \rightarrow A_i^c$ in the above equation, we 
get
\[
\bigcap_{i\in I}A_i = \left(\bigcup_{i \in I}A_i^c\right)^c \Rightarrow
\left(\bigcap_{i \in I}A_i\right)^c = \bigcup_{i \in I}A_i^c
\]

\item[5.] $\Omega = \{T^i H T^j H\;|\; i \ge 0, j \ge 0\}$.

\item[6.] Let $\Omega = \{0, 1, \ldots\}$. Let, if possible, there be a 
uniform probability distribution on it. That is, if $A, B$ are finite subsets of
$\Omega$ then $P(A) = P(B)$ if $|A| = |B|$. If $A$ and $B$ are singleton sets 
and if $P(A) = k$ then $P(\Omega) = \infty$. 

\item[7.] Let $A_1, A_2, \ldots$ be events. We will use induction on $n$.
Let us start with the base case of $n = 2$. Then,
\[
P(\cup_{i=1}^2 A_i) = P(A_1) + P(A_2) - P(A_1 \cap A_2) \le \sum_{i=1}^2 P(A_i).
\]
Assume that the hypothesis is true for all $n \le k$ and consider.
\[
P(\cup_{i=1}^{k+1} A_i) = P(A_{k+1} + \cup_{i=1}^k A_i) \le P(A_{k+1}) + 
\sum_{i=1}^k P(A_i)
\]
by induction hypothesis. Therefore,
\[
P(\cup_{i=1}^{k+1} A_i) \le \sum_{i=1}^{k+1} P(A_i)
\]
so that the hypothesis is true for all $k \in \son$.

\item[8.] Given that $P(A_i) = 1$ so that $P(A_i^c) = 0$ and
\[
P(\cup_{i \ge 1} A_i^c \le \sum_{i \ge 1} P(A_i^c) = 0,
\]
where we used the previous problem. Therefore,T
\[
P(\cup_{i \ge 1} A_i^c) = 0.
\]
Now,
\[
P(\cap_{i \ge 1}A_i) = 1 - P(\cup_{i \ge 1}A_i^c) = 1 - 0 = 1.
\]

\item[9.] Let $P(B) > 0$ for a certain event $B$. Then $P(\varnothing|B)
= P(\varnothing \cap B)/P(B) = 0$; 
\[
P(A^c|B) = \frac{P(A^c \cap B}{P(B)}
\]
Since $B$ is a disjoint union of $A \cap B$ and $A^c \cap B$, $P(B) = P(A \cap B)
+ P(A^c \cap B)$ so that
\[
P(A^c|B) = 1 - \frac{P(A \cap B)}{P(B)} = 1 - P(A|B).
\]
If $A_1, A_2, \ldots$ are pairwise disjoint sets then,
\[
P(\cup_{k \ge 1}A_k | B) = \frac{P((\cup_{k\ge 1} A_k) \cap B)}{P(B)}
= \frac{P(\cup_{k\ge 1} (A_k \cap B))}{P(B)}
\]
Since the sets $A_k \cap B$ are also pairwise disjoint,
\[
P(\cup_{k \ge 1}A_k | B) = \sum_{k \ge 1}\frac{P(A_k \cap B)}{P(B)} = 
\sum_{k \ge 1}P(A_k | B).
\]

\item[10.] Solved in the Python notebook for this chapter.

\item[11.] $P(A \cap B) = P(A)P(B) = (1 - P(A^c))(1 - P(B^c)) = 1 - P(A^c) - 
P(B^c) + P(A^c)P(B^c) = P(A) + P(B) - 1 + P(A^c)P(B^c)$ so that,
\begin{eqnarray*}
P(A^c)P(B^c) &=& 1 - P(A) - P(B) - P(A \cap B) \\
 &=& 1 - P(A \cup B) \\
 &=& P((A \cup B)^c) \\
 &=& P(A^c \cap B^c)
\end{eqnarray*}

\item[12.] Let $G_1$ and $G_2$ be the events that the first side is green and the
second side is green. Then $G_1 \cap G_2$ is the event that the first card is 
chosen and hence $P(G_1 \cap G_2) = 1/3$. $P(G_1) = 1/2$ because half the faces
are green and any one could have been chosen. Therefore,
\[
P(G_2 | G_1) = \frac{P(G_1 \cap G_2)}{P(G_1)} = \frac{1/3}{1/2} = \frac{2}{3}.
\]

\item[13.] The sample space is of the form $\{H^nT, T^nH \;|\; n \ge 1\}$. The
probability of $n + 1$ tosses before seeing both sides is 
\[
\frac{1}{2^{n+1}} + \frac{1}{2^{n+1}} = \frac{1}{2^n}.
\]
For $n + 1 = 3$, the probability is $1/4$.

\item[14.] Let $B$ denote a blue-eyed child and $\bar{B}$ denote the event if the
child's eyes are not blue.
\begin{enumerate}
\item Probability of at least two blue eyed children is the sum of probability of
all three blue-eyed children, $1/4^3$ and the probability of any two children
being blue eyed,
\[
\binom{3}{2}\left(\frac{1}{4}\right)^2\frac{3}{4} = 3\times\frac{1}{16}\times\frac{3}{4}
= \frac{9}{64}
\]
so that the required probability is $10/64$.

\item[15.] Let $B_{\ge 2}$ be the event that at least two children have blue eyes 
and $Y$ denote the event that the youngest child is blue-eyed. Then $P(Y) = 1/4$.
Further, $P(B_{\ge 2} \cap Y) = P(BBB) + P(\bar{B}BB) + P(B\bar{B}B) = 1/4^3 +
(3/4)(1/4)(1/4) + (1/4)(3/4)(1/4) = 7/64$. Therefore,
\[
P(B_{\ge 2} | Y) = \frac{7/64}{1/4} = \frac{7}{16}.
\]

\item[17.] $P(ABC) = P(A|BC)P(BC) = P(A|BC)P(B|C)P(C)$.

\item[18.] Let, if possible $P(A_i | B) < P(A_i)$ for all $i = 1, \ldots, k$. 
Since $A_1, \ldots, A_k$ partition $\Omega$,
\[
\sum_{i=1}^n P(A_i|B) < \sum_{i=1}^k P(A_i) = 1.
\]
This violates the axioms of probability for the measure $P(\cdot|B)$.
\end{enumerate}

\item[19.] Given that $P(M) = 0.3, P(W) = 0.5, P(L) = 0.2$ and that $P(V|M) =
0.65, P(V|W) = 0.82, P(V|L) = 0.5$. Note that $P(V|M) + P(V|W) + P(V|L) \ne 1$
and that is ok. We are asked to find $P(W|V)$. Using Bayes' theorem,
\[
P(W|V) = \frac{P(V|W)P(W)}{P(V|M)P(M) + P(V|W)P(W) + P(V|L)P(L)} = 0.5816
\]

\item[20.] Given that $P(H|C_1) = 0, P(H|C_2) = 0.25, P(H|C_3) = 0.5, P(H|C_4)
= 0.75, P(H|C5) = 1$.
\begin{enumerate}
\item $P(C_i|H) = P(C_i \cap H)/P(H) = P(H|C_i)P(C_i)/P(H)$. The denominator is
$P(H) = P(H|C_1)P(C_1) + P(H|C_2)P(C_2) + P(H|C_3)P(C_3) + P(H|C_4)P(C_4) +
P(H|C_5)P(C_5) = 2.5/5 = 0.5$. Therefore, $P(C_1|H) = 0, P(C_2|H) = 
(0.25 \times 0.2) /0.5 = 0.1, P(C_3|H) = (0.5 \times 0.2)/0.5 = 0.2, P(C_4|H) =
(0.75 \times 0.2)/0.5 = 0.3, P(C_5|H) = (1 \times 0.2)/0.5 = 0.4$. We check that 
\[
\sum_{k=1}^5 P(C_k|H) = 0 + 0.1 + 0.2 + 0.3 + 0.4 = 1.
\]

\item $P(H_2|H_1) = P(H_2 \cap H_1)/P(H_1)$. $P(H_1)$ is indeed $P(H)$ in the
previous step. It is $0.5$. The two coin tosses are independent, therefore, 
$P(H_2) = P(H_1)$.

\item $P(C_i|B_4) = P(B_4|C_i)P(C_i)/P(B_4)$, where
\[
P(B_4) = \sum_{j=1}^5 P(B_4|C_j)P(C_j).
\]
Now, $P(B_4|C_1) = 0, P(B_4|C_2) = 27/64, P(B_4|C_3) = 1/16, P(B_4|C_4) = 3/64,
P(B_4|C_5) = 0$ so that $P(B_4) = (34/64)(1/5)$. Therefore,
\[
P(C_i|B_4) = \frac{P(B_4|C_i)P(C_i)}{(34/64)(1/5)},
\]
as $P(C_i) = 1/5$,
\[
P(C_i|B_4) = \frac{64}{34}P(B_4|C_i).
\]
so that $P(C_1|B_4) = 0, P(C_2|B_4) = (64/34)(27/64), P(C_3|B_4) = (64/34)(1/16),
P(C_4|B_4) = (64/34)(3/64), P(C_5|B_4) = 0$. We confirm that,
\[
\sum_{i=1}^5 P(C_i|B_4) = 1.
\]
\end{enumerate}
\item[21.] Solved in the Python notebook for this chapter.

\item[22.] Solved in the Python notebook for this chapter.

\item[23.] Solved in the Python notebook for this chapter.
\end{enumerate}
\end{document}
